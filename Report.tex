% !TeX spellcheck = en_US
% !TeX program = pdflatex

\documentclass{article}

\usepackage{amsmath}
\usepackage{url}
\usepackage{graphicx}
\usepackage{float}
\usepackage{amsthm}
\usepackage[a4paper, margin=1in]{geometry}
\usepackage{amssymb}
\usepackage{enumerate}

\theoremstyle{definition}
\newtheorem{definition}{Definition}[section]
\newtheorem{problem}{Problem}
\newtheorem{example}{Example}[section]

\theoremstyle{plain}
\newtheorem{theorem}{Theorem}[section]
\newtheorem{lemma}{Lemma}[section]
\newtheorem{corollary}{Corollary}[theorem]

\theoremstyle{remark}
\newtheorem*{remark}{Remark}

\renewcommand{\Re}{\operatorname{Re}}
\renewcommand{\Im}{\operatorname{Im}}
\newcommand{\reals}{\mathbb{R}}

\title{Report on Characteristic Functions}
\author{Fu Tianwen \and Yao Chaorui \and Zhao Feng}

\begin{document}
\maketitle

\section{Propose of Presentation}
In ESTR2002, it is quite common to deal with the sum of independent random variables, or approximate sums of random variables by assuming independence. However, it is pretty difficult and tedious to handle these kind of problems with convolution. With simple knowledge of complex numbers, \textbf{Characteristic Functions} can be very helpful by turning convolution into multiplication.\\
In the presentation, we give a brief introduction about the definition, background and properties related to characteristic functions. A brief proof of \textit{Central Limit Theorem} is also given. However, we did not give any proof for the properties nor the detailed and rigorous proof of CLT because we wanted to keep only the intuition rather than the tedious details of analysis.\\
To improve our presentation quality, we prepared a handout for our fellow classmates to read during the presentation, and put all the related mathematical statements on the slide. However, since the handout is posted too late, few classmates have them on hand during the presentation; statements on the slide also make the presentation going too fast compared with handwriting them on the whiteboard. Above would be most mistakes in the reflection of our presentation.\\
Below shows the most important findings, which are almost the same as the notes given in Piazza.
\section{Definition and Properties}


\medskip

\bibliographystyle{ieeetr}
\bibliography{bibliography}
\end{document}